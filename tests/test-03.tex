\usemodule [arabic-verse]

%% test stretch option

\setuparabicverse
	[stretch=no,
	 before={\setupinterlinespace[4ex]}]

\showframe
\starttext

\startarabicverse [distance=5em]
	فما أَنْسَ مِ الأشياءِ لا أَنْسَ مَوْقِفِي ; وموقفَها وَهْنًا بقارعة النخل
	فلمّا تواقَفْنَا عرفتُ الذي بها ; كمثل الذي بي حَذْوَكَ النعلَ بالنع
	فقُلْنَ لها هذا عِيشاءٌ وأهلُنا ; قَرِيبٌ أَلَمّا تَسْأَمِي مَرْكَبَ البغل
	فقالت فما شِئْتُنّ قُلْنَ لها انزلي ; فلَلأَرضُ خيرٌ من وقوفٍ على رَحْلِ
	فأقبَلْنَ أمثالَ الدُّمَى فاكتَنَفْنَها ; وكُلٌّ يُفَدِّي بالموّة والأهل
	نُجُومٌ دَرَارِيٌّ تكنَّفْنَ صورةً ; من البدر وافتْ غيرَ هُوجٍ ولا ثُجْلِ
	فسلَّمْتُ واستأنستُ خِيفةَ أن يَرَى ; عدوٌّ مكانِي أو يرى كاشِحٌ فعلي
	فقالت وأَلْقتْ جانبَ السِّتْر إنما ; معي فتحدَّثْ غيرَ ذي رِقْبةٍ أهلي
	فقلتُ لها ما بي لهم من ترقُّبٍ ; ولكنَّ سِري ليس يحملُه مِثلي
	فلما اقتصرْنا دونهَن حديثَنا ; وهُنّ طَبِيباتٌ بحاجة ذي التَّبل
	عرَفْنَ الذي نَهْوَى فقُلْنَ ائذَنِي انا ; نَطُفْ ساعةً في بَرْدِ ليلٍ وفي سهْلِ
	فقالت فلا تَلْبَثْنَ قُلْنَ تحدَّثِي ; أتيناكِ وانْسبْنَ انسيابَ مها الرملِ
	وقُمْنَ وقد أَفْهمْنَ ذا اللُّبِّ أنما ; أَتَيْنَ الذي يَأْتِينَ من ذاك من أجْلِي
\stoparabicverse

\stoptext
